\documentclass[12pt,leqno]{amsart}
\pagestyle{plain}
\usepackage{latexsym,amsmath,amssymb}
%\usepackage[notref,notcite]{showkeys}
\usepackage{amsmath}
\usepackage{amsfonts}
\usepackage{geometry}
\usepackage{graphicx}
\graphicspath{ {images/} }
\usepackage{amssymb}

\usepackage{geometry}
\usepackage{graphicx}
\graphicspath{ {images/} }



\setlength{\oddsidemargin}{1pt}
\setlength{\evensidemargin}{1pt}
\setlength{\marginparwidth}{30pt} % these gain 53pt width
\setlength{\topmargin}{1pt}       % gains 26pt height
\setlength{\headheight}{1pt}      % gains 11pt height
\setlength{\headsep}{1pt}         % gains 24pt height
%\setlength{\footheight}{12 pt} 	  % cannot be changed as number must fit
\setlength{\footskip}{24pt}       % gains 6pt height
\setlength{\textheight}{650pt}    % 528 + 26 + 11 + 24 + 6 + 55 for luck
\setlength{\textwidth}{460pt}     % 360 + 53 + 47 for luck



\def\dsp{\def\baselinestretch{1.35}\large
\normalsize}
%%%%This makes a double spacing. Use this with 11pt style. If you
%%%%want to use this just insert \dsp after the \begin{document}
%%%%The correct baselinestretch for double spacing is 1.37. However
%%%%you can use different parameter.


\def\U{{\mathcal U}}









\begin{document}





\centerline{\bf Homework 2 for Math 1530}
\centerline{Zhen Yao}







\bigskip


\medskip

\noindent
{\bf Problem 15.}
Find $\sup A$ and $\inf A$, where
$$
A=\left\{\frac{n^2+2n-3}{n+1}:\
n=1,2,3,\ldots\right\}\, .
$$

\begin{proof}
We have
\begin{align*}
    \frac{n^2+2n-3}{n+1} & = \frac{(n+1)^2 - 4}{n+1} =  (n+1) - \frac{4}{n+1}
\end{align*}
\hspace*{2em}Substituting $n+1$ as $t > 1$, we can define a function $f(t) = t - 4/t$. Then, we have
\begin{align*}
    f'(t) = 1 + \frac{4}{t^2} > 0
\end{align*}
which means that $f(t)$ is increasing as $t \rightarrow \infty$. Thus, we have $\sup A = +\infty$. \\
\hspace*{2em}On the other hand, since $f(t)$ is increasing, we can have $\inf A = f(2) = 0$.
\end{proof}


\noindent
{\bf Problem 16.}
Prove that is $A,B\subset\mathbb{R}$ are bounded and non-empty, then
$$
\sup(A+B)=\sup A +\sup B,
\quad
\text{where}
\quad
A+B=\{x+y:\, x\in A,\ y\in B\}.
$$
\begin{proof}
(1) First, we set $M = \sup A$ and $N = \sup B$. Then for $\forall x \in A$ and  $\forall y \in B$, we have $x + y \leq M = N$. Thus, $M + N$ is an upper bound of $A+B$.\\
\hspace*{2em}(2)Since $M = \sup A$ and $N = \sup B$, then for $\forall \varepsilon > 0$, $\exists x \in A$ such that $x > M - \frac{\varepsilon}{2}$ and $\exists y \in B$ such that $y > N - \frac{\varepsilon}{2}$. Thus, we have
\begin{align*}
    x + y > M + N - \varepsilon
\end{align*}
which means $M + N$ is the least upper bound of $A + B$. The proof is complete.
\end{proof}


\noindent
{\bf Problem 17.}
Prove that is $A,B\subset (0,\infty)$ are bounded and non-empty, then
$$
\sup(A\cdot B)=\sup A \cdot\sup B,
\quad
\text{where}
\quad
A\cdot B=\{xy:\, x\in A,\ y\in B\}.
$$
\begin{proof}
(1)First, we set $M = \sup A$ and $N = \sup B$. Then for $\forall x \in A$ and  $\forall y \in B$, we have $x \cdot y \leq M \cdot N$ since $A, B \subset (0, \infty)$. Thus, $MN$ is an upper bound of $A \cdot B$.\\
\hspace*{2em}(2)Second, for $\forall \varepsilon > 0$, $\exists x \in A$ such that $x > M - \frac{\varepsilon}{2N}$, and $\exists y \in B$ such that $y > N - \frac{\varepsilon}{2M}$. Thus, we have
\begin{align*}
    x \cdot y & > \left( M - \frac{\varepsilon}{2N} \right) \left( N - \frac{\varepsilon}{2M} \right) \\
    & > MN  - \frac{\varepsilon}{2} - \frac{\varepsilon}{2} + \frac{\varepsilon^2}{MN} \\
    & > MN  - \varepsilon
\end{align*}
which means that $MN$ is a least upper bound of $a \cdot B$. The proof is complete.
\end{proof}


\noindent
{\bf Problem 18.}
Prove that if $\lim_{n\to\infty} a_n = \infty$, then
$$
\lim_{n\to\infty} \frac{a_1+a_2+\ldots+a_n}{n}=\infty.
$$
\begin{proof}
Since $\lim_{n\to\infty} = \infty$, then $\forall M \in \mathbb{R}$, there exist an $n_0$ such $\forall n > 2n_0, a_n > 2M$, then we have 
\begin{align*}
    \frac{a_1 + a_2 + \cdots + a_{n_0}}{n} & > \frac{a_1 + \cdots + a_n_0 + 2M(n - n_0)}{n} \\
    & = 2M - \frac{2n_0 M}{n} + \frac{a_1 + \cdots + a_{n_0}}{n}\\
    & > 2M - M = M
\end{align*}
Then the proof is complete.
\end{proof}


\noindent
{\bf Problem 19.}
Prove that if $\lim_{n\to\infty} a_n = g\in\mathbb{R}$ and $a_n>0$ for all $n$, then
$$
\lim_{n\to\infty} \sqrt[n]{a_1a_2\cdots a_n} = g.
$$
\begin{proof}
(1)Since $\lim_{n\to\infty} a_n = g$, we can know that there exists $M > 0$ such that for all $n$
\begin{align*}
    | a_n - g | \leq M \Rightarrow g - M \leq a_n \leq g + M
\end{align*}
Also, based on the definition of limit, we can know that for $\forall \varepsilon > 0$, $\exists n_0$ such that $\forall n > n_0$, $| a_n - g | < \varepsilon$. Using arithmetic-geometric mean inequality, we have
\begin{align*}
    \sqrt[n]{a_1 \cdots a_n} & \leq \frac{a_1 + \cdots + a_n}{n} \\
    \sqrt[n]{a_1 \cdots a_n} - g & \leq \frac{a_1 + \cdots + a_n}{n} - g 
\end{align*}
Since we already know that if $\lim_{n\to\infty} a_n = g$, then $\lim_{n\to\infty} \frac{a_1 + \cdots + a_n}{n} = g$, thus we have 
\begin{align*}
    \sqrt[n]{a_1 \cdots a_n} - g < \varepsilon
\end{align*}
\hspace*{2em}(2)Now we want to prove $- \varepsilon < \sqrt[n]{a_1 \cdots a_n} - g$. And we already know $\forall \varepsilon > 0$, $\exists n_0$ such that $\forall n > n_0$, $| a_n - g | < \varepsilon \Rightarrow g-\varepsilon < a_n < g+\varepsilon$, thus we have
\begin{align*}
    (a_1 \cdots a_{n_0-1})(g-\varepsilon)^{n-n_0+1} & \leq a_1 \cdots a_n \leq (a_1 \cdots a_{n_0-1})(g+\varepsilon)^{n-n_0+1} \\
    (g-M)^{n_0-1} (g-\varepsilon)^{n-n_0+1} & \leq a_1 \cdots a_n \leq (g+M)^{n_0-1} (g+\varepsilon)^{n-n_0+1}
\end{align*}
which yields
\begin{align*}
    (g^{n_0-1} - (n_0-1)g^{n_0-2}M + \cdots) & (g^{n-n_0+1} - (n-n_0+1)g^{n-n_0}\varepsilon + \cdots) \leq a_1 \cdots a_n \\
    & g^n - (n-n_0+1)g^{n-1}\varepsilon \leq a_1 \cdots a_n
\end{align*}
Now we just have to change the condition that $\forall \varepsilon > 0$, $\exists n_0$ such that $\forall n > n_0$, $ | a_n - g | < \varepsilon/ \left( g^{n-1}(n-n_0+1) \right)$, then we have $g^n - \varepsilon < a_1 \cdots a_n$. Then we have $\lim_{n\rightarrow \infty} a_1 \cdots a_n = g^n$, which gives us $\lim_{n\to\infty} \sqrt[n]{a_1a_2\cdots a_n} = g$.
\end{proof}

\noindent
{\bf Problem 20.}
Prove that the sequence defined by
$$
a_1=0, \qquad \text{$a_{n+1}=\sqrt{6+a_n}$ for $n\geq 1$}
$$
is convergent and find its limit.

\noindent
{\bf Remark.}
It is natural to denote the limit of this sequence by
$$
\sqrt{6+\sqrt{6+\sqrt{6+\ldots}}}
$$
\begin{proof}
We assume the sequence is convergent and its limit is $g$. Then we have 
\begin{align*}
    a_{n+1} & = \sqrt{6 + a_n}, a_0 = 0 \\
    \Rightarrow g & = \sqrt{6 + g} \\
    \Rightarrow g & = 3
\end{align*}
Now we have to prove $3$ is the limit of this sequence. \\
\hspace*{2em}(1)Since $a_{n+1} & = \sqrt{6 + a_n} \Rightarrow a_{n+1} > a_n$, this sequence is increasing. \\
\hspace*{2em}(2)Now we need to prove this sequence is bounder above. Assume that $a_n \leq 3$, then we have $a_{n+1} \leq \sqrt{6 + 3} = 3$, which means this sequence is bounded above and the limit is $3$.
\end{proof}


\noindent
{\bf Problem 21.}
Prove that
$$
2\cos\left(\frac{\pi}{2^{n+1}}\right)=\underbrace{\sqrt{2+\sqrt{2+\sqrt{2+\ldots+\sqrt{2}}}}}_{\text{$n$ {\rm square roots}}}.
$$
\begin{proof}
Based on the previous problem, we can know that $\lim \sqrt{2+\sqrt{2+\sqrt{2+\ldots+\sqrt{2}}}} = 2$. And as $n \rightarrow \infty$, $ \lim_{n\rightarrow \infty} 2\cos\left(\frac{\pi}{2^{n+1}}\right) = 2 \cos 0 = 2$. The proof is complete.
\end{proof}

\noindent
{\bf Problem 22.}
Find the limit
$$
\lim_{n\to\infty} 2^n\underbrace{\sqrt{2-\sqrt{2+\sqrt{2+\sqrt{2+\ldots+\sqrt{2}}}}}}_{\text{$n$ square roots}}.
$$
That is not a typo. We have one ``$-$'' and the rest are ``$+$'' signs.
\begin{proof}
Substituting $\sqrt{2+\sqrt{2+\sqrt{2+\ldots+\sqrt{2}}}}$ with $2 \cos \left(\frac{\pi}{2^{n}}\right)$ and set the limit of above sequence as $A$, we have 
\begin{align*}
    A &= \lim_{n\to\infty} 2^n \sqrt{2 - 2 \cos \left( \frac{\pi}{2^{n}} \right)} \\ 
    & = \lim_{n\to\infty} 2^n \sqrt{4 \sin^2 \frac{\pi}{2^{n+1}} } \\
    & = \lim_{n\to\infty} 2^{n+1} \sin \frac{\pi}{2^{n+1}} \\
    & = \pi
\end{align*}
\end{proof}


\noindent
{\bf Problem 23.}
Find the limit
$\displaystyle
\lim_{n\to\infty}
\frac{n}{e^{1+\frac{1}{2}+\cdots +\frac{1}{n}}}\, .
$
\begin{proof}
Take $\ln$ on this sequence and we have 
\begin{align*}
    \lim_{n\to\infty} \ln \left(\frac{n}{e^{1+\frac{1}{2}+\cdots +\frac{1}{n}}} \right) & = \lim_{n\to\infty} \ln n - (\ln e + \ln e^{\frac{1}{2}} + \cdots + \ln e^{\frac{1}{n}}) \\
    & = - \gamma
\end{align*}
which means $\lim_{n\to\infty}
\frac{n}{e^{1+\frac{1}{2}+\cdots +\frac{1}{n}}} = e^{- \gamma}$.
\end{proof}


\noindent
{\bf Problem 24.}
Find the limit
$\displaystyle \lim_{n\to\infty} \sin\left(2\pi\sqrt{n^2+1}\right)$.
\begin{proof}
\begin{align*}
    \lim_{n\to\infty} \sin \left(2\pi\sqrt{n^2+1}\right) & = \lim_{n\to\infty} \sin\left(2\pi n \sqrt{1 + \frac{1}{n^2}}\right) \\
    & = \lim_{n\to\infty} \sin\left(2\pi n\right) \\
    & = 0
\end{align*}

\end{proof}

\noindent
{\bf Problem 25.}
Prove that the sequence $\sin n$ is divergent.
\begin{proof}
Suppose that $\sin n$ is convergent and the limit is $g$. Then we have $\lim \sin(2n) = g \Rightarrow 2 \sin (n) \cos (n) = g$. Since $\sin^2(n) + \cos^2(n) = 1$, we have $\cos n = \sqrt{1 - g^2}$. Using $\lim \sin(2n) = g \Rightarrow 2 \sin (n) \cos (n) = g$, we have 
\begin{align*}
    2 g \sqrt{1-g^2} = g \Rightarrow g = -1,0,1
\end{align*}
Also we have $\cos (2n) = \cos^2(n) - \sin^2(n) = 1 - 2g^2$, and $\cos(2n) = \sqrt{1 - \sin^2(2n)} = \sqrt{1 - g^2}$, then
\begin{align*}
    1 - 2g^2 = \sqrt{1 - g^2} \Rightarrow g = 0
\end{align*}
Now consider $\sin(n+1) = g$, which gives $\sin n \cos 1 + \cos n \sin 1 = g$, then we have 
\begin{align*}
    g \cos 1 + \sin 1 = g
\end{align*}
where $g \neq 0$. This contradicts with above result. So $\sin n$ is divergent.
\end{proof}


\noindent
{\bf Problem 26.}
Prove that the sequence
$$
\left(1+\frac{1}{n}\right)^{n+1}
$$
is decreasing.
\begin{proof}
(1)Set $f(n) = \left(1+\frac{1}{n}\right)^{n+1} = e^{\ln \left(1+\frac{1}{n}\right)^{n+1}} = \exp \left((n+1) \ln \left(1+\frac{1}{n}\right) \right)$, then we set a function 
\begin{align*}
    f(x) = \exp \left((x+1) \ln \left(1+\frac{1}{x}\right) \right)
\end{align*}
where $x\in [1,\infty)$, and 
\begin{align*}
    f'(x) & = f(x) \left(\ln \left(1+\frac{1}{x}\right) - \frac{x+1}{x^2+x} \right) \\
    & = f(x) \left(\ln \left(1+\frac{1}{x}\right) - \frac{1}{x} \right)
\end{align*}
Now we need to determine the value of $\left(\ln \left(1+\frac{1}{x}\right) - \frac{1}{n} \right) = g(x)$, then we have
\begin{align*}
    g'(x) & = -\frac{1}{x^2 + x} + \frac{1}{x^2} > 0, x \geq 1
\end{align*}
which means $g(x)$ is increasing on $(1,\infty)$. Also, $\lim_{x\rightarrow \infty} g(x) = 0$, which implies that $f'(x) < 0$ for $x \in [1, \infty)$. Then we can know that $f(x)$ is decreasing on $[1,\infty)$, substituting $x$ by $n$ shows that the sequence is decreasing. The proof is complete. \\
\hspace*{2em}(2)Another better approach is: 
\begin{align*}
    \frac{f(n)}{f(n+1)} & = \frac{\left(1 + \frac{1}{n}\right)^{n+1}}{\left(1 + \frac{1}{n+1}\right)^{n+2}} \frac{1}{1 + \frac{1}{n+1}} \\
    & = \left(1 + \frac{1}{n^2 + 2n} \right)^{n+1} \frac{1}{1 + \frac{1}{n+1}} \\
    & \geq \left(1 + \frac{n+1}{n^2 + 2n} \right) \frac{1}{1 + \frac{1}{n+1}} \\
    & \geq \left(1 + \frac{n+1}{n^2 + 2n + 1} \right) \frac{1}{1 + \frac{1}{n+1}} \\
    & = 1
\end{align*}
The third step follows from $(1+x)^n \geq 1 + nx$. Then we know $f(n)\geq f(n+1)$, which implies it is a decreasing sequence.
\end{proof}

\end{document}

