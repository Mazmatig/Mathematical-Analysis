\documentclass[12pt,leqno]{amsart}
\pagestyle{plain}
\usepackage{latexsym,amsmath,amssymb}
%\usepackage[notref,notcite]{showkeys}
\usepackage{amsmath}
\usepackage{amsfonts}
\usepackage{geometry}
\usepackage{graphicx}
\graphicspath{ {images/} }
\usepackage{amssymb}

\usepackage{geometry}
\usepackage{graphicx}
\graphicspath{ {images/} }



\setlength{\oddsidemargin}{1pt}
\setlength{\evensidemargin}{1pt}
\setlength{\marginparwidth}{30pt} % these gain 53pt width
\setlength{\topmargin}{1pt}       % gains 26pt height
\setlength{\headheight}{1pt}      % gains 11pt height
\setlength{\headsep}{1pt}         % gains 24pt height
%\setlength{\footheight}{12 pt} 	  % cannot be changed as number must fit
\setlength{\footskip}{24pt}       % gains 6pt height
\setlength{\textheight}{650pt}    % 528 + 26 + 11 + 24 + 6 + 55 for luck
\setlength{\textwidth}{460pt}     % 360 + 53 + 47 for luck



\def\dsp{\def\baselinestretch{1.35}\large
\normalsize}
%%%%This makes a double spacing. Use this with 11pt style. If you
%%%%want to use this just insert \dsp after the \begin{document}
%%%%The correct baselinestretch for double spacing is 1.37. However
%%%%you can use different parameter.


\def\U{{\mathcal U}}










\begin{document}





\centerline{\bf Homework 3 for Math 1530}
\centerline{Zhen Yao}


\bigskip


\noindent
{\bf Problem 27.}
Let $a_1, a_2, a_3,\ldots>0$. Prove that if
$$
\lim_{n\to\infty} n\, \Big( \frac{a_n}{a_{n+1}} - 1\Big) >1\, ,
$$
then the series $a_1+a_2+a_3+\ldots$ converges.

\begin{proof}
Since $\lim_{n\to\infty} n \Big( \frac{a_n}{a_{n+1}} - 1\Big) >1$, then there exists a $r_1$ such that $\lim_{n\to\infty} n \Big( \frac{a_n}{a_{n+1}} - 1\Big) >r_1>1$. Then there exists an $N_1>0$, such that for $\forall n>N_1$, $\frac{a_n}{a_{n+1}}>1+\frac{r_1}{n}$. \\
\hspace*{3em}We take $r_2$ such that $1<r_2<r_1$. And we consider function $f(x)=1+r_1x-(1+x)^{r_2}$, which satisfies $f(0)=0$. Also, $f'(x)=r_1-r_2(1+x)^{r_2-1}>0$ in a small neighborhood of $x=0$. Then there exists an $N_2>0$ such that for $\forall n>N_2$, we have
\begin{align*}
    & \frac{a_n}{a_{n+1}}> 1+\frac{r_1}{n} > \left(1+\frac{1}{n} \right)^{r_2} = \frac{(n+1)^{r_2}}{n^{r_2}} \\
    \Rightarrow & (n+1)^{r_2} a_{n+1} < n^{r_2}a_n
\end{align*}
as $x$ substituted by $\frac{1}{n}$. Then for $n>N_2$, we have 
\begin{align*}
    a_n < \frac{N_2^{r_2}a_{N_2}}{n^{r_2}}
\end{align*}
By comparison test, $\sum^\infty_{k=1} a_k$ converges since $r_2>1$ and $\sum^\infty_{n=1} 1/n^{r_2}$.
\end{proof}

\medskip

\noindent
{\bf Problem 28.}
Provide an example of a convergent series
$a_1+a_2+a_3+\ldots$, where $a_n>0$, $n=1,2,3,\ldots$ such that
the limit $\lim_{n\to\infty} \frac{a_{n+1}}{a_n}$ does not exist.

\begin{proof}
We already know that series $\frac{1}{2} + \frac{1}{2^2} + \frac{1}{2^3} \cdots = 1$, which is convergent. Now we rearrange this series as
\begin{align*}
    a_1 = \frac{1}{2^2}, a_2 = \frac{1}{2}, a_3 = \frac{1}{2^4}, a_4 = \frac{1}{2^3}, a_5 = \frac{1}{2^6}, \cdots
\end{align*}
by substituting the positions between $2n$th and $(2n-1)$th. Then we have 
\begin{align*}
    & \lim_{n\to\infty} \frac{a_{n+1}}{a_n} = 2, n\;\text{is odd} \\
    & \lim_{n\to\infty} \frac{a_{n+1}}{a_n} = \frac{1}{8}, n\;\text{is even}
\end{align*}
which means the limit does not exist.
\end{proof}

\medskip

\noindent
{\bf Problem 29.}
Prove that there is a sequence of positive integers
$n_1<n_2< n_3 <\ldots$ such that the sequence
$a_k=\sin n_k$ converges.

\begin{proof}
Based on Bolzano-Weierstrass Theorem, we can know that bounded sequence has a convergent subsequence. Also, $\sin n$ is dense in $[-1,1]$, then there exists a subsequence that converges to any value in $[-1,1]$. Suppose we want a subsequence that converges to $g \in [-1,1]$. First, for $\forall \varepsilon > 0$, there exists $n_1$ such that $\| \sin n_1 - g\| < \varepsilon$. Then, starting from $n_1$, we could find a $n_2 > n_2$ such that  $\| \sin n_2 - g\| < \varepsilon$ since $\sin n$ is dense in $[-1,1]$. Repeating this process, and we can find $n_1 < n_2 < n_3 < \cdots$ such that $\lim_{n\to\infty} a_k = \sin n_k = g \in [-1,1]\setminus \{0\}$.
\end{proof}

\medskip



\noindent
{\bf Problem 30.}
Prove that the series
$$
\sum_{n=3}^\infty
\frac{1}{n (\log n) (\log \log n)^p}
$$
diverges if $0<p\leq 1$ and converges if $p>1$.

\begin{proof}
Based on Cauchy condensation test, the convergence of $\sum_{n=1}^\infty a_n$ is equivalent to the convergence of $\sum_{n=0}^\infty 2^n a_{2^n}$. Then we only need to consider $\sum_{n=2}^\infty 2^n a_{2^n}$ in this case, we have
\begin{align*}
    \sum_{n=2}^\infty 2^n a_{2^n} & = \sum_{n=2}^\infty 2^n \frac{1}{2^n (\ln 2^n)(\ln \ln 2^n)^p} \\
    & = \sum_{n=2}^\infty \frac{1}{(\ln 2^n)(\ln (n \ln 2))^p} \\
    & = \frac{1}{\ln 2} \sum_{n=2}^\infty \frac{1}{n(\ln n + \ln (\ln 2))^p} \\
\end{align*}
we denote this sum by $A$. And we have
\begin{align*}
   \frac{1}{\ln 2} \sum_{n=2}^\infty \frac{1}{n(\ln n)^p} \leq  A \leq \frac{1}{\ln 2} \sum_{n=2}^\infty \frac{1}{n(\frac{1}{2}\ln n)^p} = \frac{2^p}{\ln 2} \sum_{n=2}^\infty \frac{1}{n(\ln n)^p}
\end{align*}
since $\frac{1}{2}\ln n < \ln n + \ln (\ln 2) < \ln n$, for $n > 4$. And we already know that $\sum_{n=2}^\infty \frac{1}{n(\ln n)^p}$ converges if $p>1$, and diverges if $0 < p \leq 1$. So $A$ converges if $p>1$, and diverges if $0 < p \leq 1$. 
\end{proof}

\medskip


\noindent
{\bf Problem 31.}
Prove that if the series $a_1+a_2+a_3+\ldots$ converges, where $a_n>0$,
$n=1,2,3,\ldots$, then the series
$$
\sum_{n=1}^\infty \frac{\sqrt{a_n}}{n}
\quad
\mbox{converges.}
$$

\begin{proof}
We have $\frac{\sqrt{a_n}}{n}\leq \frac{1}{2}\left(a_n+\frac{1}{n^2}\right)$, since $\left(\sqrt{a_n}-\frac{1}{n}\right)^2=a_n-\frac{2\sqrt{a_n}}{n}+\frac{1}{n^2}\geq 0$. Thus, we have 
\begin{align*}
    \sum_{n=1}^\infty \frac{\sqrt{a_n}}{n} \leq \frac{1}{2}\sum_{n=1}^\infty \left(a_n + \frac{1}{n^2}\right)
\end{align*}
Then the sequence converges by comparison test.
\end{proof}

\medskip

\noindent
{\sc Definition.} Let
$a_1, a_2, a_3,\ldots>0$.
We define the infinite product by
$$
\prod_{n=1}^\infty a_n = \lim_{n\to\infty} a_1 a_2\ldots a_n\, .
$$
We say that the infinite product {\em converges} if the limit is finite and
{\em positive}. If the limit does not exist, equals $0$ or $\infty$ then
we say that the product {\em diverges}.

\medskip

\noindent
{\bf Problem 32.}
Prove that if $a_n>0$, $n=1,2,\ldots$, then
the product $\prod_{n=1}^\infty (1+a_n)$ converges if and only if
the series $\sum_{n=1}^\infty a_n$ converges.
{\bf Hint:} {\em You can use the inequality $e^x\geq 1+x$ without proving it.}

\begin{proof}
Denote $\prod_{n=1}^\infty (1+a_n)$ by $A$.\\
(1)If the series $\sum_{n=1}^\infty a_n$ converges, we have
\begin{align*}
    \ln A & = \ln (1 + a_1) + \cdots + \ln (1 + a_n) \\
    & \leq \ln e^{a_1} + \cdots + \ln e^{a_n} \\
    & = \sum^\infty_{n=1}  a_n
\end{align*}
Since $\sum_{n=1}^\infty a_n$ converges, then $\ln A$ converges. Thus, $A$ converges since $\log$ function is continuous.\\
(2)If $\prod_{n=1}^\infty (1+a_n)$ converges, we can prove following inequality by induction
\begin{align*}
    1 + \sum^N_{n=1}a_n \leq \prod^N_{n=1}(1+a_n)
\end{align*}
For $N=1$, $1+a_1 \leq 1+a_1$, so it holds. Assume it also holds for $N=k$, then for $N=k+1$, we have
\begin{align*}
    1 + \sum^{N+1}_{n=1}a_n & \leq \prod^N_{n=1}(1+a_n) + a_n \\
    & \leq \prod^N_{n=1}(1+a_n) + \prod^N_{n=1}(1+a_n) a_{n+1} \\
    & = \prod^{N+1}_{n=1}(1+a_n)
\end{align*}
So We can know 
\begin{align*}
    \sum^\infty_{n=1}a_n \leq \prod^\infty_{n=1}(1+a_n) - 1
\end{align*}
which implies that $\sum^\infty_{n=1}a_n$ converges.
\end{proof}

\medskip


\noindent
{\bf Problem 33.}
Prove that if $0<a_n<1, n=1,2,\ldots$, then the series
$\sum_{n=1}^\infty a_n$ converges if and only if the series
$\sum_{n=1}^\infty a_n/(1-a_n)$ converges.

\begin{proof}
(1)If $\sum_{n=1}^\infty a_n$ converges, then it implies that $\lim_{n\to\infty} a_n= 0$ then $\forall \varepsilon > 0$, $\exists N_1 > 0$ such that $\forall n > N_1$, $a_n < \varepsilon$. Since it is true for arbitrary $\varepsilon > 0$, then there exist an $N_2 > 0$, such that $a_n < \varepsilon < \frac{1}{2}$. Also, since $\sum_{n=1}^\infty a_n$ converges, then $\forall \varepsilon > 0$, $\exists N_3 > 0$, such that for $\forall n > N_3, \forall m > 0$, $\left| a_n + \cdots + a_{n+m} \right| < \varepsilon$. Now we set $N = \max \{N_1, N_2, N_3 \}$, we have
\begin{align*}
    \left|\frac{a_n}{1 - a_n} + \cdots + \frac{a_{n+m}}{1 - a_{n+m}}\right| & \leq 2(a_n + \cdots + a_{n+m}) \leq 2 \varepsilon
\end{align*}
since $a_n < \varepsilon < \frac{1}{2}$ for $n > N$. Then we proved that $\sum_{n=1}^\infty a_n/(1-a_n)$ converges. \\
\hspace*{2em}(2)If $\sum_{n=1}^\infty a_n/(1-a_n)$ converges, then we have 
\begin{align*}
    \sum_{n=1}^\infty |a_n| < \sum_{n=1}^\infty a_n/(1-a_n)
\end{align*}
since $0 < a_n < 1$ for $\forall n$. So $\sum_{n=1}^\infty a_n$ converges.
\end{proof}

\medskip


\noindent
{\bf Problem 34.}
Prove that if $0<a_n<1$, then the product $\prod_{n=1}^\infty (1-a_n)$ converges
if and only if the series $\sum_{n=1}^\infty a_n$ converges.

\begin{proof}
(1)If $\sum_{n=1}^\infty a_n$ converges, then we have 
\begin{align*}
    \ln \left(\prod^\infty_{n=1} (1 - a_n) \right) = \sum^\infty_{n=1} \ln (1 - a_n) \leq \sum^\infty_{n=1} a_n 
\end{align*}
since $\ln (1 - x) < x, 0 < x < 1$. Also, $\log$ function is continuous and we have that $\prod^\infty_{n=1} (1 - a_n)$ converges. \\
\hspace*{2em}(2)If $\prod_{n=1}^\infty (1-a_n)$ converges, we can know $\prod_{n=1}^\infty 1/(1-a_n)$ also converges, since $0 < a_n < 1$ which means $1 - a_n \neq 0$. Using inequality $e^{-x} > 1- x$, we have $e^x < \frac{1}{1 - x}$. Then we have
\begin{align*}
    \sum^\infty_{n=1} a_n < \ln \left(\prod^\infty_{n=1} \frac{1}{1 - a_n} \right)
\end{align*}
Then $\sum^\infty_{n=1} a_n$ converges. The proof is complete.
\end{proof}





\end{document}

