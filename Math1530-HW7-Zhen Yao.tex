\documentclass[12pt,leqno]{amsart}
\pagestyle{plain}
\usepackage{latexsym,amsmath,amssymb}
%\usepackage[notref,notcite]{showkeys}
\usepackage{amsfonts}
\usepackage{geometry}
\usepackage{graphicx}
\graphicspath{ {images/} }

\setlength{\oddsidemargin}{1pt}
\setlength{\evensidemargin}{1pt}
\setlength{\marginparwidth}{30pt} % these gain 53pt width
\setlength{\topmargin}{1pt}       % gains 26pt height
\setlength{\headheight}{1pt}      % gains 11pt height
\setlength{\headsep}{1pt}         % gains 24pt height
%\setlength{\footheight}{12 pt} 	  % cannot be changed as number must fit
\setlength{\footskip}{24pt}       % gains 6pt height
\setlength{\textheight}{650pt}    % 528 + 26 + 11 + 24 + 6 + 55 for luck
\setlength{\textwidth}{460pt}     % 360 + 53 + 47 for luck

\newtheorem{theorem}{Theorem}



\def\dsp{\def\baselinestretch{1.35}\large
\normalsize}
%%%%This makes a double spacing. Use this with 11pt style. If you
%%%%want to use this just insert \dsp after the \begin{document}
%%%%The correct baselinestretch for double spacing is 1.37. However
%%%%you can use different parameter.


\def\U{{\mathcal U}}

\begin{document}

\centerline{\bf Homework 7 for Math 1530}
\centerline{Zhen Yao}

\bigskip

\noindent
{\bf Problem 73.}
For $n\geq 2$ define $f_n: [0,1]\to[0,1]$ by
$$
f_n(x) = \left\{\begin{array}{lcl}
	nx & & \mbox{if}\ 0\leq x\leq \frac{1}{n} \\
	\frac{n}{n-1}(1-x) & & \mbox{if}\ \frac{1}{n} \leq x \leq 1 \end{array}\right.
$$
Show that $\sum_{n=2}^{\infty} \left[f_n(x)\right]^n$ converges pointwise on $[0,1]$ to a function $f(x)$ that is continuous on $(0,1]$, but that the improper integral $\int_0^1 f(x)\,\mathit{dx}$ diverges
(the integral is improper at $0$).
\begin{proof}
If $x = 0$ or $x = 1$, then $f_n(x) = 0, \forall n\in\mathbb{N}$, then $\sum^\infty_{n=2}[f_n(x)]^n$ converges to $f(x) = 0$.For fixed $x\in(0,1)$, we can know that there exists $N > 0$ such that $\frac{1}{N}\leq x$. Then we can have 
\begin{align*}
    \sum^\infty_{n=2}[f_n(x)]^n & = \sum^N_{n=2}(nx)^n + \sum^\infty_{n=N+1} \left(\frac{n}{n-1}\right)^n (1-x)^n
\end{align*}
The first term of the right hand side is finite since it is the summation of finite real numbers. Now we need to show that the second term of the right hand side converges. Since for $\forall n \geq N+1$, we can know that $0< 1 - x < \frac{n}{n-1}$, then there exists a $\gamma$ such that $\frac{1}{1-x}>\gamma>\frac{n-1}{n}$. Then we have 
\begin{align*}
    \sum^\infty_{n=N+1} \left(\frac{n}{n-1}\right)^n (1-x)^n \leq \sum^\infty_{n=N+1}\gamma^n(1-x)^n
\end{align*}
and we already know that $\gamma (1-x) < 1$, and with the fact that $\sum (\gamma (1-x))^n$ converges, we can conclued that the second term also converges. Thus, $\sum^\infty_{n=2}[f_n(x)]^n$ converges pointwise on [0,1] to a function $f(x)$.\\
\hspace*{3em} For $\int_0^1 f(x) \mathit{dx}$, we have 
\begin{align*}
    \int_0^1 f(x) \mathit{dx} & = \int_0^1 \sum_{n=2}^{\infty} \left[f_n(x)\right]^n \mathit{dx} \\
    & = \sum_{n=2}^{\infty} \int_0^{1/n} n^n x^n \mathit{dx} + \sum_{n=2}^{\infty} \int_{1/n}^1 \left(\frac{n}{n-1}\right)^n(1-x)^n \mathit{dx} \\
    & = \sum_{n=2}^{\infty} \frac{1}{n(n+1)} + \sum_{n=2}^{\infty} \frac{n-1}{n(n+1)} \\
    & = \sum_{n=2}^{\infty} \frac{1}{n+1} \\
    & = \sum_{n=3}^{\infty} \frac{1}{n}
\end{align*}
As we know that $\sum^\infty_{n=1}\frac{1}{n}$ diverges, then we have $\sum_{n=3}^{\infty} \frac{1}{n}$ also diverges. Thus, $\int_0^1 f(x) \mathit{dx}$ diverges.
\end{proof} 

\medskip

\noindent
{\bf Problem 74.}
Prove that $\displaystyle\lim_{x\to\infty}e^{-x^2}\int_0^x e^{t^2}\, dt = 0$.
\begin{proof}
With $\lim_{x\infty}\frac{1}{e^{x^2}} = \infty$, we can use L'Hospital's rule, which yields
\begin{align*}
    \lim_{x\to\infty}\frac{\int_0^x e^{t^2} dt}{e^{x^2}} & = \lim_{x\to\infty}\frac{e^{x^2}}{2x e^{x^2}} = \lim_{x\to\infty}\frac{1}{2x} = 0
\end{align*}
The proof is complete.
\end{proof}

\medskip

\noindent
{\bf Problem 75.}
For $n$ a positive integer, put
$$
t_n=\frac{1}{2n+1}-\frac{1}{2n+2}+\frac{1}{2n+3}-\frac{1}{2n+4}+\cdots+\frac{1}{4n-1}-\frac{1}{4n}.
$$
Find, with proof, the limit $\lim_{n\to\infty} nt_n$.
\begin{proof}
We have 
\begin{align*}
    \lim_{n\to\infty}nt_n = & n \left(\frac{1}{2n+1}-\frac{1}{2n+2}+\frac{1}{2n+3}-\frac{1}{2n+4}+\cdots+\frac{1}{4n-1}-\frac{1}{4n} \right) \\
    = & n \left(\frac{1}{2n+1}+\frac{1}{2n+2}+\frac{1}{2n+3}-\frac{1}{2n+4}+\cdots+\frac{1}{4n-1}+\frac{1}{4n} \right)\\
    & - 2n \left(\frac{1}{2n+2} + \frac{1}{2n+4} + \cdots + \frac{1}{4n}\right) \\
    = & n \int_0^2 \frac{1}{2+x}dx - n \int^1_0 \frac{1}{1+x}dx \\
    = & n\ln 4 - n\ln 2 - n\ln 2 \\
    = & 0
\end{align*}
The proof is complete.
\end{proof}

\medskip

\noindent
{\bf Problem 76.}
Evaluate the limit $\displaystyle\lim_{n\to\infty}\int_1^2\frac{nx^2}{1+n^2x^4}\, dx$.
\begin{proof}
We have 
\begin{align*}
    \int_1^2\frac{nx^2}{1+n^2x^4} dx & \leq \int_1^2\frac{4n}{1+n^2x} dx \\
    & = 4n \left. \left(\frac{1}{n^2}\ln(1+n^2x) \right) \right|_1^2 \\
    & = \frac{4}{n} \ln \left(\frac{2n^2 + 1}{n^2 + 1}\right)
\end{align*}
which implies $\lim_{n\to\infty}\int_1^2\frac{nx^2}{1+n^2x^4} dx \leq \lim_{n\to\infty}\frac{4}{n} \ln \left(\frac{2n^2 + 1}{n^2 + 1}\right) = 0$. Thus,  $\lim_{n\to\infty} \int_1^2\frac{nx^2}{1+n^2x^4} dx = 0$.
\end{proof}

\medskip

\noindent
{\bf Problem 77.}
Prove that
$$
\lim_{t\to 1^-}\int_0^t\left(\int_0^t\frac{dx}{1-xy}\right)\, dy=\sum_{n=1}^\infty\frac{1}{n^2}.
$$
\begin{proof}
We have 
\begin{align*}
    \int_0^t\left(\int_0^t\frac{dx}{1-xy}\right) dy & = \int_0^t \left. \left(-\frac{1}{y}\ln (1-xy)\right)\right|^t_0 dy \\
    & = \int_0^t \left(-\frac{1}{y}\ln (1 - ty) + \frac{1}{y}\ln 1\right) dy \\
    & = - \int_0^t \frac{1}{y}\ln (1 - ty) dy
\end{align*}
Now using Taylor theorem to expand $\ln (1 - ty)$ at point $y=0$, then we get 
\begin{align*}
    - \int_0^t \frac{1}{y}\ln (1 - ty) dy & = - \int_0^t \frac{1}{y}\left(-ty - \frac{t^2y^2}{2} - \frac{t^3y^3}{3} - \cdots \right) dy \\
    & = \int_0^t \left(t + \frac{t^2y}{2} + \frac{t^3y^2}{3} + \cdots \right) dt \\
    & = t^2 + \frac{t^4}{2^2} + \frac{t^6}{3^2} + \cdots + \frac{t^{2n}}{n^2} +\cdots
\end{align*}
Thus, we have $$\lim_{t\to 1^-} - \int_0^t \frac{1}{y}\ln (1 - ty) dy = \lim_{t\to 1^-} t^2 + \frac{t^4}{2^2} + \frac{t^6}{3^2} + \cdots + \frac{t^{2n}}{n^2} +\cdots = \sum^\infty_{n=1}\frac{1}{n^2}$$ 
The proof is complete.
\end{proof}

\medskip

\noindent
{\bf Problem 78.}
Let $f:[0,1]\to\mathbb{R}$ be a continuous function. Prove that
$$
\lim_{n\to\infty} n\int_0^1 x^nf(x)\, dx=f(1).
$$
\begin{proof}
It suffice to show that 
\begin{align*}
    \lim_{n\to\infty} (n+1)\int_0^1 x^nf(x)\, dx=f(1)
\end{align*}
Then we have 
\begin{align*}
    \left|(n+1)\int_0^1 x^nf(x)\, dx - f(1)\right| & = \left|(n+1)\int_0^1 x^nf(x)\, dx - f(1)\int_0^1 (n+1)x^n\, dx\right| \\
    & = \left|(n+1)\int_0^1 x^n \left(f(x)-f(1)\right)\, dx\right| \\
    & \leq (n+1)\int_0^1 x^n \left| \left(f(x)-f(1)\right)\right|\, dx
\end{align*}
Since $f$ is continuous on $[0,1]$, then for $\forall \varepsilon > 0$, there exists $0 < \delta < 1$ such that for $x\in[\delta, 1]$, $\left|f(x) - f(1)\right| < \varepsilon$. Then we have 
\begin{align*}
    (n+1)\int_0^1 x^n \left| \left(f(x)-f(1)\right)\right|\, dx = &\,  (n+1)\int_0^\delta x^n \left| \left(f(x)-f(1)\right)\right|\, dx \\
    & + (n+1)\int_\delta^1 x^n \left| \left(f(x)-f(1)\right)\right|\, dx \\
    & \leq \int^\delta_0(n+1)\delta^n \cdot 2M\, dx + \int^1_\delta (n+1)x^n \varepsilon\, dx \\
    & < 2M\cdot (n+1)\delta^n + \varepsilon \xrightarrow{n\to\infty} \varepsilon
\end{align*}
where $M = \sup_{x\in[0,1]}|f(x)|$. Then we have 
\begin{align*}
    \lim_{n\to\infty}\left|(n+1)\int_0^1 x^nf(x)\, dx - f(1)\right| < \varepsilon
\end{align*}
and this holds for arbitrary $\varepsilon$, then 
\begin{align*}
    \lim_{n\to\infty}\left|(n+1)\int_0^1 x^nf(x)\, dx - f(1)\right| = 0
\end{align*}
The proof is complete.
\end{proof}

\medskip

\noindent
{\bf Problem 79.}
Let $f:[0,1]\to\mathbb{R}$ be continuously differentiable on $[0,1]$ and satisfy $f(1)=0$. Prove that
$$
\int_0^1 |f(x)|^2\, dx\leq 4\int_0^1 x^2|f'(x)|^2\, dx.
$$
\begin{proof}
Without losing generality, we assume $f(x)>0$. Since if $f(x)$ takes values that some are $>0$ and some $<0$, then we can consider it in differential intervals where $f$ does not change sign. Then, with $f(1) = 0$, we have 
\begin{align*}
    \int_0^1 |f(x)|^2 dx & = \left. xf(x)^2 \right|^1_0 - \int^1_0 2x f(x) f'(x) dx \\
    & = 0 + \int^1_0 2x f(x) (-f'(x)) dx \\
    & \leq 2 \left(\int^1_0 f^2(x) dx \right)^{\frac{1}{2}} \left(\int^1_0 x^2 |f'(x)|^2 dx \right)^{\frac{1}{2}}
\end{align*}
then we divide both sides of the equation and we get 
\begin{align*}
    \left(\int_0^1 |f(x)|^2 dx \right)^{\frac{1}{2}} & \leq 2 \left(\int^1_0 x^2 |f'(x)|^2 dx \right)^{\frac{1}{2}}\\
    \Rightarrow \int_0^1 |f(x)|^2 dx & \leq 4 \int^1_0 x^2 |f'(x)|^2 dx
\end{align*}
The proof is complete.
\end{proof}

\medskip

\noindent
{\bf Problem 80.}
Suppose $f:[1,\infty)\to\mathbb{R}$ is continuous and $\lim_{x\to\infty} xf(x)=1$. Prove that
$$
\lim_{t\to\infty}\frac{1}{t}\int_1^{e^t}f(x)\, dx=1.
$$
\begin{proof}
Using L'Hospital's rule, we can have 
\begin{align*}
    \lim_{t\to\infty}\frac{1}{t}\int_1^{e^t}f(x) dx & = \lim_{t\to\infty}\frac{f(e^t)(e^t)'}{1} \\
    & = \lim_{t\to\infty} e^t f(e^t) \\
    & = 1
\end{align*}
In the last step, we used $\lim_{x\to\infty}xf(x)=1$ and $\lim_{t\to\infty}e^t = \infty$. The proof is complete.
\end{proof}

\medskip

\noindent
{\bf Problem 81.}
Prove that $\displaystyle\lim_{n\to\infty}\int_0^1n\ln\left(1+\frac{x}{n}\right)\, dx=\frac{1}{2}$.
\begin{proof}
We have 
\begin{align*}
    \lim_{n\to\infty}\int_0^1n\ln\left(1+\frac{x}{n}\right)dx & = n \left[\int^1_0\ln (n+x)dx - \int^1_0\ln ndx\right] \\
    & = n \int^1_0\ln (n+x)dx - n \\
    & = n \left(\left. x\ln (n+x)\right|^1_0 - \int^1_0 \frac{x}{n+x}dx\right) - n \\
    & = n(n+1)\ln \left(1+\frac{1}{n}\right) - n
\end{align*}
With Taylor expansion of $\ln x = x - \frac{x^2}{2} + \frac{x^3}{3} - \cdots$, we have 
\begin{align*}
    \lim_{n\to\infty}n(n+1)\ln \left(1+\frac{1}{n}\right) - n & = \lim_{n\to\infty} n(n+1)\left(\frac{1}{n} - \frac{1}{2n^2} + \cdots\right) - n \\
    & = \lim_{n\to\infty} (n+1) - \frac{n^2+n}{2n^2} + \cdots - n \\
    & = \frac{1}{2}
\end{align*}
The proof is complete.
\end{proof}

\noindent
{\bf Problem 82.}
Evaluate the integral
$$
I=\int_2^4\frac{\sqrt{\ln(9-x)}}{\sqrt{\ln(9-x)}+\sqrt{\ln(3+x)}}\, dx.
$$
{\bf Hint:}{\em This is an easy problem if you look for symmetry.}
\begin{proof}
We can know that $9-x$ and $3+x$ are symmetry as to $6$, where $2\leq x \leq 4$, then we can change variable that $x = 3 + t, t\in[-1,1]$. Then the integral become 
\begin{align*}
    I & = \int_{-1}^1\frac{\sqrt{\ln(6+t)}}{\sqrt{\ln(6+t)}+\sqrt{\ln(6+t)}} dt \\
    & = \int_{-1}^1 \frac{1}{2}dt \\
    & = 1
\end{align*}
\end{proof}

\medskip

\noindent
{\bf Problem 83.}
Show that for every positive integer $n$,
$$
\left(\frac{2n-1}{e}\right)^{\frac{2n-1}{2}}<1 \cdot 3\cdot 5\cdots(2n-1)<
\left(\frac{2n+1}{e}\right)^{\frac{2n+1}{2}}
$$
{\bf Hint:} {\em Estimate the integral of $\ln x$ by Riemann sums both from above and from below.}
\begin{proof}
Take $\ln$ function to both sides, and we have the left side as $\frac{1}{2}x\ln x - \frac{1}{2}x$, while $x = 2n - 1$, this is equal to $\int^{2n-1}_1\ln x dx$. And the right side will become $\frac{1}{2}x\ln x - \frac{1}{2}x$, while $x = 2n+1$, and this is equal to $\int^{2n+1}_1\ln x dx$. The middle one is actually equal to one half of Riemann sum of integral of $\ln x$ from $1$ to $2n$, since $\int^{2n}_1 \ln x dx = 2 (f(1) + f(3) + f(5) + \cdots + f(2n-1)) = 2 (\ln 1 + \ln 3 + \cdots + \ln(2n-1))$. Since $\int^{2n-1}_1\ln x dx<\int^{2n}_1\ln x dx<\int^{2n+1}_1\ln x dx$, then we have 
\begin{align*}
    \frac{2n-1}{2}\ln \left(\frac{2n-1}{e}\right) < \ln(1 \cdot 3\cdot 5\cdots(2n-1)) < \frac{2n+1}{2}\ln \left(\frac{2n+1}{e}\right)
\end{align*}
which is exactly the inequality above.
\end{proof}












\end{document}
